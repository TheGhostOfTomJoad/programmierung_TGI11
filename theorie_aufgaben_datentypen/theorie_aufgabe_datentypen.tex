\documentclass[DIV = 11]{scrartcl}
\usepackage[subpreambles=true]{standalone}
\usepackage{import}

\usepackage[]{microtype}

\usepackage[
typ=ab,
fach=Informatik,
lerngruppe=TGI11,
farbig,
namensfeldAnzeigen,
%datumAnzeigen,
seitenzahlen = auto,
loesungen = seite,
module= {Papiertypen}
]{schule}
\ifoot{% TODO: \usepackage{graphicx} required

		\includegraphics[width=0.3\linewidth]{GHSE-Logo}

}
\usepackage{fourier-otf}
\title{Datentypen}
\usepackage{amsthm, amsmath}
\newtheorem{theorem}{Gesetz}
\newtheorem{regel}{Regel}
\newtheorem*{bsp}{Beispiel}
\newtheorem*{schreibweise}{Schreibweise}
\renewcommand*{\proofname}{Herleitung}
\AtBeginDocument{\renewcommand\proofname{Herleitung}}
\usepackage{minted}
\usepackage{amsmath,fontspec,newunicodechar}


\NewDocumentCommand{\skull}{}{%
  \text{\skullfont\symbol{"1F571}}%
}
\newunicodechar{🕱}{\skull}
\begin{document}

\begin{aufgabe}
Kann der Ausdruck fehlerfrei ausgewertet werden? Falls ja, bestimme den  Wert und den Typ des Ausdrucks.\\
\hinweis{Der Typ eines Ausdrucks ist der Typ des Ergebnisses.}
\begin{teilaufgaben}

\teilaufgabe
\begin{minted}{Kotlin}
1 + "2"
\end{minted}

\teilaufgabe
\begin{minted}{Kotlin}
"1".toInt() + 2
\end{minted}

\teilaufgabe
\begin{minted}{Kotlin}
"1".toInt() + 2.toInt()
\end{minted}

\teilaufgabe
\begin{minted}{Kotlin}
"1".toInt() + 2.toString()
\end{minted}

\teilaufgabe
\begin{minted}{Kotlin}
"1".toInt() + "2".toInt()
\end{minted}

\teilaufgabe
\begin{minted}{Kotlin}
1.toString() + "2".toInt()
\end{minted}
\end{teilaufgaben}
\end{aufgabe}

\begin{aufgabe}
Welche globalen Variablen sind nach der Ausführung definiert. Welchen Typ und welchen Wert haben diese?


\begin{minted}{Kotlin}
val a = 3

val b = 4

val c = "a + b"

val d = a + b

val e = "a.toString() + b.toString()"

val f = a.toString() + b.toString()

val g = (a + b).toString()
\end{minted}

\end{aufgabe}


\end{document}
