% !TeX TS-program = xelatex


% Full instructions available at:
% https://github.com/pcafrica/focus-beamertheme

\documentclass{beamer}
\usetheme{focus}

\title{Informatik SG J1}
\subtitle{Organisatorisches}
\author{Herr Hertenstein}
%\titlegraphic{\includegraphics[scale=1.25]{focus-logo.pdf}}
%\institute{GHSE}
%\date{dd mm yyyy}
\date{}
% Footline info is printed only if [numbering=fullbar].
%\footlineinfo{Custom footline text}

\begin{document}
    \begin{frame}
        \maketitle
    \end{frame}
    
    % Use starred version (e.g. \section*{Section name})
    % to disable (sub)section page.
    \section{Organisatorisches}
	
	\begin{frame}[plain]
	\begin{itemize}
	\item 6 Klassenarbeiten in zwei Fächern
	\item Schriftlich : $\frac{4}{5}$ 
	\item Mündlich : $\frac{1}{5}$ 
	\end{itemize}
	\end{frame}

		\section{Regeln}

	\begin{frame}[plain]
	\begin{itemize}
	\item Pünktlichkeit
	\item Arbeitsmaterial/Tablet mitnehmen
	\item Keine Spiele 
	\item Kein Smartphone
	\item Respektvoller Umgang
	\item Feedback
 	\item Eigenständigkeit
	\item Zusammenarbeit
	\end{itemize}
\end{frame}

	\section{Inhalte}
		\begin{frame}[plain]
	\begin{itemize}
	\item Imperative Programmierung
	\item Objektorientierte Programmierung
	\item vieles baut aufeinander auf
	\item Hausaufgaben
	\end{itemize}
	\end{frame}

	\section{Tastaturen}

%	\section{Programmieren (in Python)}
%    \begin{frame}[plain]{Plain frame}
%        	\begin{itemize}
%	\item Nützlich für die Auswertung von Daten 
%	\item Experimente mit Psychopy erstellen 
%	\item Kurse in vielen Studiengängen
%	\item großer Bonus bei Bewerbungen
%	\item Übung notwendig!
%	\item macht Spaß!
%	\end{itemize}
%    \end{frame}
%    
%	
%
%\section{Thonny installieren}
%\begin{frame}[plain]
% \begin{enumerate}
%\item \url{https://thonny.org/}
%\item Windows 64 Bit Python
%\item Herunterladen
%\item Mit Doppelklick installieren 
%\end{enumerate}
%	
%	  \end{frame}
%
%
%\section{Alternative}
%\begin{frame}[plain]
%\begin{itemize}
%\item \url{https://www.programiz.com/python-programming/online-compiler/}
%\item Tastaturen: z.B. Logitech K380, HP350, Logitech MK 370, Logitech Slim Folio iPad (muss zum Modell passen). 
%\item Thonny ist klar besser und wird auch in der Klausur verwendet
%\end{itemize}
%
%
%\end{frame}

 \section{10-Fingersystem}
	\begin{frame}[plain]
\url{https://www.typingclub.com/tipptrainer}
\end{frame}


\section{EDV-Zugänge}

\section{Nextcloud}

\begin{frame}[plain]
\begin{itemize}
\item Link in der App eintragen
\item \url{https://idp.em-ghse.logoip.de}
\end{itemize}
\end{frame}


\section{Vorstellungsrunde}    
 
\end{document}
