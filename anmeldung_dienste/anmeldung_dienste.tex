\documentclass[]{scrartcl}
\usepackage[
typ=ib,
fach=Informatik,
lerngruppe=TGI11,
farbig,
%namensfeldAnzeigen,
%datumAnzeigen,
seitenzahlen = auto,
%loesungen = folgend,
module= {Papiertypen}
]{schule}
\usepackage{hyperref}
\usepackage{fourier-otf}
\usepackage{url}
\usepackage{simpleicons}
\ifoot{% TODO: \usepackage{graphicx} required
	
	\includegraphics[width=0.35\linewidth]{GHSE-Logo}
	
}
\title{Anmeldung-Dienste}

\begin{document}

\section{Anonyme Mailadresse}
Wenn du nicht deine eigene Mail-Adresse verwenden willst, kannst du dir eine neue anonyme Mail-Adresse erstellen. Ansonsten kannst du diesen Schritt überspringen. 
\section{Notwendige Schritte}
\begin{enumerate}
\item Melde dich auf Github an: \url{https://github.com/}

\item Melde dich mit deinem Github-Account bei Jetbrains an:
 \url{https://account.jetbrains.com/login}. Klicke dafür auf das \simpleicon{github}

\item Logge dich mit deinem Jetbrains-Account bei Datalore ein: \url{https://datalore.jetbrains.com/notebooks}. Klicke dafür auf \texttt{Sign in with JetBrains Account},
\end{enumerate}
\section{Weitere sinnvolle Schritte}
\begin{enumerate}


\item Installiere die GitHub-App und melde dich mit deinem Account an.

\item Melde dich für eine GitHub-Education Lizenz an:
\url{https://education.github.com/pack}. Hierfür benötigst du ein Bild deines Schülerausweises.

\item Nutze den Zugang aus dem letzten Schritt um eine JetBrains Education Lizenz zu erhalten. \url{https://www.jetbrains.com/shop/eform/students}

\item Lade zuhause InteliJ herunter und installiere es \url{https://www.jetbrains.com/idea/download/} Wenn du die letzten beiden Schritte ausgeführt hast, kannst du die Ultimate-Version installieren. Ansonsten musst die die Community-Edition verwenden.

\item Speichere den Link zur neuen Nextcloud an der GHSE \url{https://nextcloud-g2.em-ghse.logoip.de/index.php/apps/files/files}. Es ist auch sinnvoll die Nextcloud-App auf dem Handy zu installieren.


\end{enumerate}


\end{document}
