\documentclass[a4paper, fontsize = 13, DIV = calc]{scrartcl}
\usepackage[
typ={ab},
fach=Informatik,
lerngruppe=SG-J1,
%loesungen=seite,
%lizenz={cc-by-nc-sa-4}
module={Aufgaben},
farbig
]{schule}
%\usepackage[libertinus]{fontsetup}
%\usepackage{tikz}
%\usepackage{ctable}
%
%\usepackage{cochineal}
\usepackage{amsmath}
\usepackage{amssymb}
\usepackage{scrlayer-scrpage}
\ifoot{% TODO: \usepackage{graphicx} required
	
	\includegraphics[width=0.4\linewidth]{GHSE-Logo}
	
}
%\usepackage{fontspec}
%\setmonofont{Monoki}
\usepackage[ngerman]{babel} 
%\usepackage{polyglossia}
%\setmainlanguage{ngerman}  % -> "Contents"
%\setmainlanguage{german}  % -> "Inhaltsverzeichnis"
%\usepackage[utf8]{inputenc}			% Set encoding
%\usepackage[T1]{fontenc}
%\usepackage[default]{fontsetup}
\usepackage[erewhon]{fontsetup}
\setmonofont{Ubuntu Mono Regular}%[Scale=0.8]
%\setmonofont{Iosevka}%[Scale=0.8]
%\usepackage{polyglossia}
%\setmainlanguage{ngerman}
\usepackage{shellesc}
\ShellEscape{pythontex \jobname.pytxcode }
%\usepackage{minted}
\usepackage{pythontex}
\usepackage{microtype}	
\usepackage[font=tiny,labelfont=bf]{caption}
\usepackage{float}
%\author{Maximilian Hertenstein}
%\date{\today}
\title{Einstieg - mehrere Alternativen}


%\usepackage{svg}
%\usepackage{ctable}
\date{}

\newcommand{\expandpyconc}[1]{\expandafter\reallyexpandpyconc\expandafter{#1}}
\newcommand{\reallyexpandpyconc}[1]{\pyconc{exec(compile(open('#1', 'rb').read(), '#1', 'exec'))}}

\newenvironment{pyconcodeblck}[1]
{\newcommand{\snippetfile}{snippet-#1.py}
	\VerbatimEnvironment
	\begin{VerbatimOut}{\snippetfile}}
	{\end{VerbatimOut}
	\expandpyconc{\snippetfile}}

\newcommand{\typesetcode}[1]{\inputpygments{python}{snippet-#1.py}}


\hypersetup{hidelinks}
%\usepackage{ctable}
%\usepackage{svg}
%\KOMAoptions{DIV=25}
%\usepackage[default]{fontsetup}
\begin{document}



\begin{pyconcodeblck}{temp}
def to_much_alcohol(amount: int) -> str:
    if amount < 10:
        return "low risk"
    else:
        return "unhealty"
	
def to_much_alcohol_better(amount: int) -> str:
    if amount < 10:
		return "low risk"
    elif amount <= 20:
		return "unhealty"
    else:
		return "very unhealty"
\end{pyconcodeblck}


\begin{aufgabe} \noindent 
Schreibe eine Funktion \mintinline{python}{to_much_alcohol}, die Integer übergeben bekommt. Dieses gibt an wie viel Gramm Alkohol eine Person pro Woche zu sich nimmt. Die Funktion gibt zurück wie ungesund der Alkoholkonsum bei dieser Menge ist. Bei weniger  als $10$ ist die Rückgabe \mintinline{python}{"low risk"}. Bei einer größeren Menge ist die Rückgabe \mintinline{python}{"unhealty"}
\end{aufgabe}
\begin{pyconsole}
to_much_alcohol(9)
to_much_alcohol(10)
to_much_alcohol(20)
to_much_alcohol(21)
\end{pyconsole}


%\begin{aufgabe}\noindent 
%Schreibe die letzte Funktion nochmal und verwende nur ein \mintinline{python}{return}-\emph{Statement}\\
%
%\hinweis{Nutze Variablen/Zuweisungen}
%\end{aufgabe}



\begin{aufgabe}\noindent 
Schreibe eine Funktion \mintinline{python}{to_much_alcohol_better}. Wenn die Person mehr als $20$ Gramm Alkohol zu sich nimmt, wird jetzt \mintinline{python}{"very unhealty"} zurückgegeben.
\end{aufgabe}
\begin{pyconsole}
to_much_alcohol_better(9)
to_much_alcohol_better(10)
to_much_alcohol_better(20)
to_much_alcohol_better(21)
\end{pyconsole}






%
%\begin{aufgabe}
%Löse die vorherige Aufgabe mit
%\begin{enumerate}
%\item Verschachtlungen
%\item \mintinline{python}{elif}
%\item ohne Verschachtlungen und ohne \mintinline{python}{elif}
%\end{enumerate}
%\end{aufgabe}


\end{document}
