\documentclass[class=scrartcl, crop=false]{standalone}
\usepackage[
typ={ab},
fach=Informatik,
lerngruppe=TGI11,
%loesungen=seite,
%lizenz={cc-by-nc-sa-4}
module={Aufgaben},
farbig
]{schule}
%\usepackage{polyglossia}
%\setmainlanguage{ngerman}
\usepackage[subpreambles=true]{standalone}
\usepackage{import}
\usepackage{fourier-otf}
\setmonofont{Ubuntu Mono Regular}[Scale=0.9]
\usepackage{shellesc}
\ShellEscape{pythontex \jobname.pytxcode }
\usepackage[
%prettyprinter=pygments, %pygopt={style=emacs}
]{pythontex}
\title{Variablen neue Werte zuweisen}

\usepackage{hyperref}


\usepackage{booktabs}

\usepackage{minted}





\newcommand{\expandpyconc}[1]{\expandafter\reallyexpandpyconc\expandafter{#1}}
\newcommand{\reallyexpandpyconc}[1]{\pyconc{exec(compile(open('#1', 'rb').read(), '#1', 'exec'))}}

\newenvironment{pyconcodeblck}[1]
{\newcommand{\snippetfile}{snippet-#1.py}
	\VerbatimEnvironment
	\begin{VerbatimOut}{\snippetfile}}
	{\end{VerbatimOut}
	\expandpyconc{\snippetfile}}

\newcommand{\typesetcode}[1]{\inputpygments{python}{snippet-#1.py}}



\begin{document}



\section{Werte von Variablen ändern}


\begin{aufgabe} \noindent




Welche Werte haben die Variablen am Ende?
\begin{teilaufgaben}
\teilaufgabe \begin{minted}[]{kotlin}
var x = 3
var y = x
x = 4 * x - x
y = x - y
\end{minted}



\teilaufgabe \begin{minted}[]{kotlin}
var x = 2
var y = 5
x = y
y = x
\end{minted}


\teilaufgabe \begin{minted}[]{kotlin}
var x = "hello"
val y = "world"
val z = x
x = y
x = z + " " + x
\end{minted}
\teilaufgabe
\begin{minted}{kotlin}
var a =  || false
val b = false && true
a = a || b
\end{minted}

\end{teilaufgaben}


\end{aufgabe}


\end{document}
