\documentclass[DIV = 11]{scrartcl}
\usepackage[subpreambles=true]{standalone}
\usepackage{import}

\usepackage[]{microtype}

\usepackage[
typ=ab,
fach=Informatik,
lerngruppe=TGI11,
farbig,
namensfeldAnzeigen,
%datumAnzeigen,
seitenzahlen = auto,
loesungen = seite,
module= {Papiertypen}
]{schule}
\ifoot{% TODO: \usepackage{graphicx} required

		\includegraphics[width=0.3\linewidth]{GHSE-Logo}

}
\usepackage{fourier-otf}
\title{Datentypen}
\usepackage{amsthm, amsmath}
\newtheorem{theorem}{Gesetz}
\newtheorem{regel}{Regel}
\newtheorem*{bsp}{Beispiel}
\newtheorem*{schreibweise}{Schreibweise}
\renewcommand*{\proofname}{Herleitung}
\AtBeginDocument{\renewcommand\proofname{Herleitung}}
\usepackage{minted}
\usepackage{amsmath,fontspec,newunicodechar}


\NewDocumentCommand{\skull}{}{%
  \text{\skullfont\symbol{"1F571}}%
}
\newunicodechar{🕱}{\skull}
\begin{document}

\begin{aufgabe}
Bestimme den Wert des Ausdrucks.
\begin{teilaufgaben}

\teilaufgabe
\begin{minted}{Kotlin}
2 * 5 < 12
\end{minted}

\teilaufgabe
\begin{minted}{Kotlin}
10 < 10
\end{minted}

\teilaufgabe
\begin{minted}{Kotlin}
8 <= 8
\end{minted}

\teilaufgabe
\begin{minted}{Kotlin}
5 < 3
\end{minted}

\teilaufgabe
\begin{minted}{Kotlin}
2 < 3
\end{minted}

\teilaufgabe
\begin{minted}{Kotlin}
5 <= 6
\end{minted}

\teilaufgabe
\begin{minted}{Kotlin}
-1 * -1 == 1
\end{minted}

\teilaufgabe
\begin{minted}{Kotlin}
"hello" + "world" != "hello world"
\end{minted}

\teilaufgabe
\begin{minted}{Kotlin}
8 > 2 * 4
\end{minted}

\teilaufgabe
\begin{minted}{Kotlin}
7 > 2 * 4
\end{minted}

\teilaufgabe
\begin{minted}{Kotlin}
10 > 3 * 3
\end{minted}

\teilaufgabe
\begin{minted}{Kotlin}
10 >= 3 * 3
\end{minted}

\teilaufgabe
\begin{minted}{Kotlin}
10 >= 5 * 2
\end{minted}

\teilaufgabe
\begin{minted}{Kotlin}
10 > 5 * 2
\end{minted}
\end{teilaufgaben}
\end{aufgabe}

\begin{aufgabe}
Welche globalen Variablen sind nach der Ausführung definiert. Welchen Typ und welchen Wert haben diese?

\begin{teilaufgaben}

\teilaufgabe \begin{minted}{Kotlin}
val a = 2 * 3

val b = 3 * 3

val c = a + 3 <= b

val d = c == true
\end{minted}

\teilaufgabe \begin{minted}{Kotlin}
val distanceInMeter = 3

val distanceInCentiMeter = 300

fun meterToMillimetes (distanceInMeter:Int): Int {
    val distanceInCentiMeter = 100 * distanceInMeter
    return 10 * distanceInCentiMeter 
}	

val x = meterToMillimetes(2) > distanceInCentiMeter 
\end{minted}

\teilaufgabe \begin{minted}{Kotlin}
fun firstIsAtLeastFiveTimesSecond(x: Int, y:Int): Bool =  x >= 5 * y

val x = 21

val y = 4

val z = firstIsAtLeastFiveTimesSecond(y, x)
\end{minted}
\end{teilaufgaben}


\end{aufgabe}


\begin{aufgabe}
Welche Werte haben die Ausdrücke!
\begin{teilaufgaben}
\teilaufgabe \begin{minted}[]{Kotlin}
3 > 2 || 4 < 3
\end{minted}

\teilaufgabe \begin{minted}[]{Kotlin}
3 > 2 || false
\end{minted}

\teilaufgabe \begin{minted}[]{Kotlin}
3 > 2 && 4 < 3
\end{minted}

\teilaufgabe \begin{minted}[]{Kotlin}
(!(3 > 2) && 4 < 3)
\end{minted}

\teilaufgabe \begin{minted}[]{Kotlin}
!(3 > 2 && 4 < 3)
\end{minted}

\end{teilaufgaben}
\end{aufgabe}

\begin{aufgabe}
Welche globalen Variablen sind nach der Ausführung definiert.
\begin{teilaufgaben}
\teilaufgabe \begin{minted}[]{Kotlin}
val x = 3

val y = 3 < 4

val z = y  &&  !(x != 3)
\end{minted}

\teilaufgabe \begin{minted}[]{Kotlin}
fun ordered(x: Int, y: Int, z: Int): Bool = x <= y and y <= z 

val z = 3

val x = 1

val y = 2

val result = ordered(z, x, y)
\end{minted}


\end{teilaufgaben}


\end{aufgabe}


\end{document}
