\documentclass[a4paper, fontsize = 13, DIV = calc]{scrartcl}
\usepackage[
typ={ab},
fach=Informatik,
lerngruppe=SG-J1,
%loesungen=seite,
%lizenz={cc-by-nc-sa-4}
module={Aufgaben},
farbig
]{schule}
%\usepackage[libertinus]{fontsetup}
%\usepackage{tikz}
%\usepackage{ctable}
%
%\usepackage{cochineal}
\usepackage{amsmath}
\usepackage{amssymb}
\usepackage{scrlayer-scrpage}
\ifoot{% TODO: \usepackage{graphicx} required
	
	\includegraphics[width=0.4\linewidth]{GHSE-Logo}
	
}
%\usepackage{fontspec}
%\setmonofont{Monoki}
\usepackage[ngerman]{babel} 
%\usepackage{polyglossia}
%\setmainlanguage{ngerman}  % -> "Contents"
%\setmainlanguage{german}  % -> "Inhaltsverzeichnis"
%\usepackage[utf8]{inputenc}			% Set encoding
%\usepackage[T1]{fontenc}
%\usepackage[default]{fontsetup}
\usepackage[erewhon]{fontsetup}
\setmonofont{Ubuntu Mono}%[Scale=0.8]
%\setmonofont{Iosevka}%[Scale=0.8]
%\usepackage{polyglossia}
%\setmainlanguage{ngerman}
\usepackage{shellesc}
\ShellEscape{pythontex \jobname.pytxcode }
%\usepackage{minted}
\usepackage{pythontex}
\usepackage{microtype}	
\usepackage[font=tiny,labelfont=bf]{caption}
\usepackage{float}
%\author{Maximilian Hertenstein}
%\date{\today}
\title{Einstiegsaufgabe Verknüpfungen}


%\usepackage{svg}
%\usepackage{ctable}
\date{}

\newcommand{\expandpyconc}[1]{\expandafter\reallyexpandpyconc\expandafter{#1}}
\newcommand{\reallyexpandpyconc}[1]{\pyconc{exec(compile(open('#1', 'rb').read(), '#1', 'exec'))}}

\newenvironment{pyconcodeblck}[1]
{\newcommand{\snippetfile}{snippet-#1.py}
	\VerbatimEnvironment
	\begin{VerbatimOut}{\snippetfile}}
	{\end{VerbatimOut}
	\expandpyconc{\snippetfile}}

\newcommand{\typesetcode}[1]{\inputpygments{python}{snippet-#1.py}}


\hypersetup{hidelinks}
%\usepackage{ctable}
%\usepackage{svg}
%\KOMAoptions{DIV=25}
%\usepackage[default]{fontsetup}
\begin{document}



\begin{pyconcodeblck}{temp}
def to_gas_station(km_to_gas_station, km_per_l, liter, flat_tire):
	return km_to_gas_station <= km_per_l * liter and not flat_tire
\end{pyconcodeblck}





\begin{aufgabe} \noindent 
Schreibe eine Funktion \mintinline{python}{to_gas_station}. Mit dieser Funktion kann man berechnen, ob man es zur nächsten Tankstelle schafft. Diese Funktion hat drei Parameter vom Typ Integer. Dabei handelt es sich um die Entfernung zur Tankstelle in Kilometern, die Kilometeranzahl, die mit einem Liter Benzin gefahren werden können und die übrige Tankfüllung in Litern. Außerdem wird ein Boolean übergeben. Dieses gibt an, ob man das Auto einen platten Reifen hat . Die Funktion gibt zurück, ob du es bis zur nächsten Tankstelle schaffst.
\begin{pyconsole}
to_gas_station(101, 50, 2, False)
to_gas_station(100, 50, 2, False)
to_gas_station(100, 50, 2, True)
\end{pyconsole}
\end{aufgabe}



\newpage



%\section{Aufgaben zum Kapitel \enquote{Typen, Werte, Ausdrücke, Operatoren}}
%
%
%\begin{aufgabe} \noindent
%Erkläre die folgenden Begriffe:
%\begin{teilaufgaben}
%\teilaufgabe Operator
%\teilaufgabe Operand
%\teilaufgabe Ausdruck
%\teilaufgabe Typ
%\teilaufgabe Typfehler
%\teilaufgabe Wert
%\teilaufgabe Statement
%\teilaufgabe Befehl
%\end{teilaufgaben}
%\end{aufgabe}
%
%\begin{aufgabe} \noindent
%Beantworte für jeden Ausdruck ob er fehlerfrei ausgewertet werden kann. Bestimme für jeden Ausdruck, der ausgewertet werden kann, den Wert und den Typ.
%\begin{teilaufgaben}
%		\teilaufgabe \mintinline{python}{1 + 2 * (1000 - 33)}
%		\teilaufgabe \mintinline{python}{"1" + "2" + "3"}
%		\teilaufgabe \mintinline{python}{"4" + 5 + "6"}
%\end{teilaufgaben}
%
%\end{aufgabe}
%
%
%\section{Aufgaben zum Kapitel \enquote{Funktionen}}
%
%\begin{aufgabe} \noindent
%	Wie kann der Ausdruck \mintinline{python}{7 + 8 + "9"} mit Funktionen zur Typkonversion angepasst werden, damit dass Ergebnis
%	\begin{teilaufgaben}
%		\teilaufgabe \mintinline{python}{24} ist?
%		\teilaufgabe \mintinline{python}{"789"} ist?
%	\end{teilaufgaben}
%\end{aufgabe}
%
%\section{Aufgaben zum Kapitel \enquote{Skriptmodus}}
%\begin{aufgabe} \noindent
%
%Schreibe ein Skript, das in einer Zeile deinen Vornamen und einer zweiten Zeile deinen Nachnamen ausgibt.
%\end{aufgabe}
%
%\section{Aufgaben zum Kapitel \enquote{Funktionen definieren}}
%
%Schreibe die Lösungen der folgenden Aufgaben in ein Skript mit dem Namen \texttt{functions\_1.py}\\
%
%
%\achtung{Verwende bei jeder Funktion die vollständige und korrekte Typannotation!}\\
%
%
%
%%\begin{aufgabe} \noindent
%%	Schreibe eine Funktion \mintinline{python}{good_morning}, die einen Namen als String übergeben bekommt und der Person auf der Konsole \enquote{Guten Morgen!} wünscht.
%%\end{aufgabe}
%
%\begin{aufgabe} \noindent
%Schreibe eine Funktion \mintinline{python}{good_morning}, die einen Namen als String übergeben bekommt und der Person auf der Konsole \enquote{Guten Morgen!} wünscht.
%\end{aufgabe}
%
%\begin{pyconsole}
%good_morning("Alan Turing")
%\end{pyconsole}
%
%\begin{aufgabe} \noindent
%\begin{teilaufgaben}
%	\teilaufgabe Schreibe eine Funktion \mintinline{python}{hours_to_minutes}, die ein Integer nimmt, das eine Zeit in Stunden angibt und die selbe Zeit in Minuten zurückgibt.
%	\teilaufgabe Schreibe eine Funktion \mintinline{python}{days_to_hours}, die ein Integer nimmt, das eine Zeit in Tagen angibt und die selbe Zeit in Stunden zurückgibt.
%	\teilaufgabe Schreibe eine Funktion \mintinline{python}{days_to_minutes}, die ein Integer nimmt, das eine Zeit in Tagen angibt und die selbe Zeit in Minuten zurückgibt. Nutze die vorherigen Teilaufgaben!
%\end{teilaufgaben}
%\end{aufgabe}
%
%\begin{pyconsole}
%hours_to_minutes(13)
%days_to_hours(2)
%days_to_minutes(1)
%\end{pyconsole}
%
%
%
%
%
%
%
%
%
%\newpage
%\section{Mehr Aufgaben}
%
%\subsection{Integer}
%
%
%
%
%
%
%\url{https://www.codewars.com/kata/58261acb22be6e2ed800003a/train/python}
%
%
%\url{https://www.codewars.com/kata/55f73be6e12baaa5900000d4/train/python}\\
%
%
%\subsection{Strings}
%
%
%
%
%
%
%\url{https://www.codewars.com/kata/544675c6f971f7399a000e79/train/python}\\
%
%
%
%\url{https://www.codewars.com/kata/5265326f5fda8eb1160004c8/train/python}\\
%
%
%
%\subsection{Booleans}
%
%
%\url{https://www.codewars.com/kata/551b4501ac0447318f0009cd/train/python}\\
%
%
%
%
%
%
%
%
%
%\subsection{Modulo}
%\url{https://www.codewars.com/kata/555086d53eac039a2a000083/train/python}\\
%
%
%
%
%
%
%
%
%
%
%
%
%\subsection{Floats}
%\url{https://www.codewars.com/kata/57a429e253ba3381850000fb/train/python}\\
%
%\section{Bedingungen}
%
%
%
%
%
%
%
%
%
%
%
%
%%\url{https://www.codewars.com/kata/57089707fe2d01529f00024a/train/python}\\
%
%
%
%
%
%
%
%
%
%
%
%
%
%
%%\url{https://www.codewars.com/kata/57e3f79c9cb119374600046b/train/python}\\
%
%
%\section{For-Schleifen}
%
%\subsection{Grundlagen}
%
%
%
%
%\url{https://www.codewars.com/kata/55caef80d691f65cb6000040/train/python}\\
%
%% \url{https://www.codewars.com/kata/528e95af53dcdb40b5000171/train/python}\\
%
%\url{https://www.codewars.com/kata/57241e0f440cd279b5000829/train/python}\\
%
%%\url{https://www.codewars.com/kata/555eded1ad94b00403000071/train/python}\\
%
%
%
%\subsection{Sehr anspruchsvoll}
%
%
%
%
%%\url{https://www.codewars.com/kata/56269eb78ad2e4ced1000013/train/python}\\
%
%
%
%
%%\url{https://www.codewars.com/kata/59a8570b570190d313000037}\\
%
%%\url{}\\
%%\url{}\\
%
%
%
%\section{While-Schleifen}
%
%\subsection{Grundlagen}
%%\url{https://www.codewars.com/kata/534d0a229345375d520006a0/train/python}\\
%
%\url{https://www.codewars.com/kata/57be674b93687de78c0001d9/train/python}\\
%
%% \url{https://www.codewars.com/kata/563b662a59afc2b5120000c6/train/python}\\
%
%
%
%\url{https://www.codewars.com/kata/58941fec8afa3618c9000184/train/python}\\
%
%
%\url{https://www.codewars.com/kata/55f2b110f61eb01779000053/train/python}\\
%
%\subsection{Sehr anspruchsvoll}
%
%\url{https://www.codewars.com/kata/588425ee4e8efb583d000088/train/python}\\
%
%
%\url{https://www.codewars.com/kata/577a6e90d48e51c55e000217/train/python}\\
%
%\url{https://www.codewars.com/kata/5286b2e162056fd0cb000c20/train/python}\\
%
%
%
%\url{https://www.codewars.com/kata/56d46b8fda159582e100001b/train/python}\\
%
%%\url{https://www.codewars.com/kata/57241e0f440cd279b5000829/train/python}\\
%
%\url{https://www.codewars.com/kata/56ba65c6a15703ac7e002075/train/python}\\
%
%
%
%
%
%
%%\url{https://www.codewars.com/kata/5a58d46cfd56cb4e8600009d/train/python}\\
%
%\url{https://www.codewars.com/kata/54c27a33fb7da0db0100040e/train/python}\\
%
%\url{https://www.codewars.com/kata/562f91ff6a8b77dfe900006e/train/python}\\
%
%
%
%\section{Variablen}
%\url{https://www.codewars.com/kata/5612e743cab69fec6d000077/train/python}\\
%
%
%\section{Strings - Iteration}
%\url{https://www.codewars.com/kata/57eae20f5500ad98e50002c5/train/python}\\
%
%\url{https://www.codewars.com/kata/596fba44963025c878000039/train/python}\\
%
%\url{https://www.codewars.com/kata/56b1f01c247c01db92000076/train/python}\\
%
%\url{https://www.codewars.com/kata/56bf3287b5106eb10f000899/train/python}\\
%
%\url{https://www.codewars.com/kata/52f3149496de55aded000410/train/python}\\
%
%%\url{}\\
%\section{Listen}
%
%\url{https://www.codewars.com/kata/52a723508a4d96c6c90005ba/train/python}\\
%
%\url{https://www.codewars.com/kata/57cc975ed542d3148f00015b/train/python}\\
%
%%\url{https://www.codewars.com/kata/511f0fe64ae8683297000001/train/python}\\
%
%%\url{https://www.codewars.com/kata/511f0fe64ae8683297000001/train/python}\\
%
%%\url{https://www.codewars.com/kata/525c1a07bb6dda6944000031/train/python}\\
%
%
%%\url{https://www.codewars.com/kata/57cc975ed542d3148f00015b/train/python}\\
%
%%\url{https://www.codewars.com/kata/5951d30ce99cf2467e000013/train/python}\\
%
%%\url{https://www.codewars.com/kata/57f781872e3d8ca2a000007e/train/python}\\
%
%%\url{https://www.codewars.com/kata/5266876b8f4bf2da9b000362}\\
%%\url{}\\


\end{document}
