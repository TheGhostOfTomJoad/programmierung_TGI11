\documentclass[a4paper, DIV = calc]{scrartcl}
\usepackage[
typ={ohne},
fach=Informatik,
lerngruppe=SG-J1,
%loesungen=seite,
%lizenz={cc-by-nc-sa-4}
module={Aufgaben},
farbig
]{schule}
\usepackage[]{standalone}
\usepackage{import}

\usepackage{graphicx}
\usepackage{fourier-otf}
\setmonofont{Ubuntu Mono Regular}[Scale=0.9]
\usepackage{shellesc}
\ShellEscape{pythontex \jobname.pytxcode }
\usepackage[]{pythontex}

\usepackage[]{csquotes}

\usepackage{amsthm}
\newtheorem*{matheaufgabe}{Mathematikaufgabe}
\newtheorem*{loesung_der_matheaufgabe}{Lösung}
\usepackage{float}
\usepackage{placeins}

\usepackage{booktabs}

\usepackage{babel}


\usepackage{minted}
\title{Aufgabensammlung Python}
%\usepackage{polyglossia}
%\setmainlanguage{ngerman}

\newcommand{\expandpyconc}[1]{\expandafter\reallyexpandpyconc\expandafter{#1}}
\newcommand{\reallyexpandpyconc}[1]{\pyconc{exec(compile(open('#1', 'rb').read(), '#1', 'exec'))}}

\newenvironment{pyconcodeblck}[1]
{\newcommand{\snippetfile}{snippet-#1.py}
	\VerbatimEnvironment
	\begin{VerbatimOut}{\snippetfile}}
	{\end{VerbatimOut}
	\expandpyconc{\snippetfile}}

\newcommand{\typesetcode}[1]{\inputpygments{python}{snippet-#1.py}}

%\usepackage[hidelinks]{hyperref}
\usepackage{scrlayer-scrpage}
\ifoot{
	
	\includegraphics[width=0.35\linewidth]{GHSE-Logo}
	
}


\begin{document}
%\maketitle
\pagenumbering{gobble}
\maketitle
%\pagenumbering{arabic}
%\thispagestyle{empty}
\pagenumbering{gobble}
\tableofcontents
%\thispagestyle{empty}
\pagebreak
\pagenumbering{arabic}


\import{sections/01_werte_und_ausdruecke/}{werte_und_ausdruecke_aufgaben}
\pagebreak
\import{sections/02_variablen/}{variablen_aufgaben}
\pagebreak
\import{sections/03_funktionen/}{funktionen_aufgaben}
\pagebreak
\import{sections/04_strings/}{strings_aufgaben}
\pagebreak
\import{sections/05_datentypen/}{datentypen_aufgaben}
\pagebreak
\import{sections/06_funktionen_und_strings/}{funktionen_und_strings_aufgaben}
\pagebreak
\import{sections/07_ein_und_ausgabe/}{ein_und_ausgabe_aufgaben}
\pagebreak
\import{sections/08_booleans/}{booleans_aufgaben}
\pagebreak
\import{sections/09_bedingte_ausführung/}{bedingte_ausführung_aufgaben}
\pagebreak
\import{sections/10_werte_von_variablen_ändern/}{werte_von_variablen_ändern_aufgaben}
\pagebreak
\import{sections/11_for_schleifen/}{for_schleifen_aufgaben}
\pagebreak
\import{sections/12_while_schleifen/}{while_schleifen_aufgaben}
\pagebreak
\import{sections/13_strings_als_container/}{strings_als_container_aufgaben}
\pagebreak
\import{sections/14_listen/}{listen_aufgaben}
\pagebreak
%\import{sections/15_listen_ändern/}{listen_ändern_aufgaben}
%\pagebreak
%\import{sections/16_floats/}{floats_aufgaben}
%\pagebreak
\end{document}
