\documentclass[class=scrartcl, crop=false]{standalone}
\usepackage[
typ={ohne},
fach=Informatik,
lerngruppe=SG-J1,
%loesungen=seite,
%lizenz={cc-by-nc-sa-4}
module={Aufgaben},
farbig
]{schule}
%\usepackage{polyglossia}
%\setmainlanguage{ngerman}
\usepackage[subpreambles=true]{standalone}
\usepackage{import}
\usepackage{fourier-otf}
\setmonofont{Ubuntu Mono Regular}[Scale=0.9]
\usepackage{shellesc}
\ShellEscape{pythontex \jobname.pytxcode }
\usepackage[
%prettyprinter=pygments, %pygopt={style=emacs}
]{pythontex}


\usepackage{hyperref}


\usepackage{booktabs}

\usepackage{minted}





\newcommand{\expandpyconc}[1]{\expandafter\reallyexpandpyconc\expandafter{#1}}
\newcommand{\reallyexpandpyconc}[1]{\pyconc{exec(compile(open('#1', 'rb').read(), '#1', 'exec'))}}

\newenvironment{pyconcodeblck}[1]
{\newcommand{\snippetfile}{snippet-#1.py}
	\VerbatimEnvironment
	\begin{VerbatimOut}{\snippetfile}}
	{\end{VerbatimOut}
	\expandpyconc{\snippetfile}}

\newcommand{\typesetcode}[1]{\inputpygments{python}{snippet-#1.py}}



\begin{document}

\begin{pyconcodeblck}{funktionen_und_strings}
def say_hello(name):
    return "Hello, " + name
    
def greet(name):
    return "Hello, " + name + " how are you doing today?"




def combine_names(firstName, lastName):
    return firstName + " " + lastName
\end{pyconcodeblck}



\section{Funktionen und Strings}


\begin{aufgabe} \noindent
Implementiere eine Funktion \mintinline{python}{say_hello}, die einen Namen als String übergeben bekommt und eine Begrüßung an die Person zurückgibt.

\begin{pyconsole}
say_hello("Ada")
say_hello("Alan Turing")
\end{pyconsole}

\noindent\url{https://www.codewars.com/kata/5625618b1fe21ab49f00001f/train/python}
\end{aufgabe}



\begin{aufgabe} \noindent
Implementiere eine Funktion \mintinline{python}{greet}, die einen Namen als String übergeben bekommt und eine freundlichere Begrüßung an die Person zurückgibt.
	
\begin{pyconsole}
greet("Ina")
greet("Grace Hopper")
\end{pyconsole}
	
\noindent\noindent\url{https://www.codewars.com/kata/55a70521798b14d4750000a4/train/python}
\end{aufgabe}


\begin{aufgabe} \noindent
Implementiere eine Funktion \mintinline{python}{combineNames}, die Vor- und Nachnamen einer Person als Strings übergeben bekommt und den vollen Namen der Person zurückgibt.
	
\begin{pyconsole}
combine_names("Alan", "Turing")
combine_names("Grace", "Hopper")
\end{pyconsole}
	
\noindent\url{https://www.codewars.com/kata/55f73f66d160f1f1db000059/train/python}
\end{aufgabe}

\end{document}
