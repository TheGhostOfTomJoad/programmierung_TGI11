\documentclass[class=scrartcl, crop=false]{standalone}
\usepackage[
typ={ab},
fach=Informatik,
lerngruppe=TGI-11,
%loesungen=seite,
%lizenz={cc-by-nc-sa-4}
module={Aufgaben},
farbig
]{schule}
%\usepackage{polyglossia}
%\setmainlanguage{ngerman}

\usepackage{fourier-otf}
\setmonofont{Ubuntu Mono Regular}[Scale=0.9]




\title{Werte und Ausdrücke}





\begin{document}
\noindent
Führe die Berechnungen in einem Terminal oder einem Jupyter-Dokument durch.
Beachte, dass immer nur der der Wert der letzten Zeile angezeigt wird.


\section{Werte und Ausdrücke}
\begin{aufgabe} \noindent 
Berechne in der Shell 
	
\begin{teilaufgaben}
\teilaufgabe Wie viele Minuten hat ein Tag?
\teilaufgabe Wie viele Minuten hat eine Woche?
\teilaufgabe Die Summe der Zahlen von 1 bis 10
\end{teilaufgaben}

\end{aufgabe}

%\begin{aufgabe} \noindent

%Berechne in der Shell
%\begin{teilaufgaben}
%\teilaufgabe $21^{2}$
%\teilaufgabe $41^{2}$
%\teilaufgabe $77^{2}$
%\teilaufgabe $123456789^{2}$
%\end{teilaufgaben}
%\end{aufgabe}

\begin{aufgabe}
\noindent
Für die englischen Längeneinheiten Fuß, Yard und Zoll(Inch) gelten:
\begin{itemize}
	\item Ein Yard sind $3$ Fuß
	\item Ein Fuß sind $12$ Zoll
\end{itemize}
Wie viel sind $\,7$ Yard in Fuß und in Zoll?
\end{aufgabe}

\begin{aufgabe}
\noindent
Finde einen Ausdruck in dem keine zwei verschiedene Rechenzeichen vorkommen und das Setzen einer Klammer zu einem anderen anderen Ergebnis führt.
\end{aufgabe}


Für die Bearbeitung der folgenden Aufgaben musst du dich mit Integer-Division auskennen.

\begin{aufgabe}
Berechne wie viele Stunden,  Minuten und Sekunden \begin{enumerate}
\item $3600$
\item $3601$
\item $4576$ 
\end{enumerate} 

Sekunden sind.
\end{aufgabe}

\end{document}
