\documentclass[class=scrartcl, crop=false]{standalone}
\usepackage[
typ={ohne},
fach=Informatik,
lerngruppe=SG-J1,
%loesungen=seite,
%lizenz={cc-by-nc-sa-4}
module={Aufgaben},
farbig
]{schule}
%\usepackage{polyglossia}
%\setmainlanguage{ngerman}
\usepackage[subpreambles=true]{standalone}
\usepackage{import}
\usepackage{fourier-otf}
\setmonofont{Ubuntu Mono Regular}[Scale=0.9]
\usepackage{shellesc}
\ShellEscape{pythontex \jobname.pytxcode }
\usepackage[
%prettyprinter=pygments, %pygopt={style=emacs}
]{pythontex}


\usepackage{hyperref}


\usepackage{booktabs}

\usepackage{minted}





\newcommand{\expandpyconc}[1]{\expandafter\reallyexpandpyconc\expandafter{#1}}
\newcommand{\reallyexpandpyconc}[1]{\pyconc{exec(compile(open('#1', 'rb').read(), '#1', 'exec'))}}

\newenvironment{pyconcodeblck}[1]
{\newcommand{\snippetfile}{snippet-#1.py}
	\VerbatimEnvironment
	\begin{VerbatimOut}{\snippetfile}}
	{\end{VerbatimOut}
	\expandpyconc{\snippetfile}}

\newcommand{\typesetcode}[1]{\inputpygments{python}{snippet-#1.py}}



\begin{document}

\begin{pyconcodeblck}{booleans}


def set_alarm(employed: bool, vacation: bool) -> bool:
    return employed and not vacation

def xor(a: bool, b: bool) -> bool:
    return (a and not b) or (not a and b)

def hero(bullets, dragons) -> bool:
   return bullets  >= 2 * dragons

def is_triangle(a, b, c):
    return a < b + c and b < a + c and c < a + b

def zero_fuel(distance_to_pump, mpg, fuel_left):
    return mpg * fuel_left >= distance_to_pump


def check_for_factor(base, factor):
    return base % factor == 0
\end{pyconcodeblck}

\section{Booleans}


\subsection{Vergleiche}
\begin{aufgabe} \noindent
Ein Held ist auf dem Weg zu einem Schatz, der von Drachen beschützt wird. Um einen Drachen zu besiegen braucht er zwei Kugeln. Hat er zu wenige Kugeln wird er von den Drachen gefressen. Implementiere eine Funktion \mintinline{python}{hero}, die zwei Parameter vom Typ \emph{Integer} hat. Der erste Parameter ist die Anzahl der Kugeln, die der Held dabei hat und der zweite Parameter ist die Anzahl de Drachen. Die Funktion soll zurückgeben, ob der Held überlebt.

\noindent\url{https://www.codewars.com/kata/59ca8246d751df55cc00014c/train/python}\\
\end{aufgabe}

\begin{pyconsole}
hero(10, 5)
hero(7, 4)
\end{pyconsole}


\begin{aufgabe} \noindent
Implementiere eine Funktion \mintinline{python}{zero_fuel}, die dir hilft, zu berechnen, ob du es mit einem Auto und dem übrigen Benzin bis zur nächsten Tankstelle schaffst. Diese Funktion hat drei Parameter vom Typ \emph{Integer}. Dabei handelt es sich um die Entfernung zur Tankstelle in Kilometern, die Kilometeranzahl, die mit einem Liter Benzin gefahren werden können und die übrige Tankfüllung in Litern. Sie gibt zurück, ob du es bis zur nächsten Tankstelle schaffst.
	
\begin{pyconsole}
zero_fuel(50, 25, 2)
zero_fuel(100, 50, 1)
\end{pyconsole}
	
\noindent\url{https://www.codewars.com/kata/5861d28f124b35723e00005e/train/python}
\end{aufgabe}
%\section{Aufgaben zum Kapitel \enquote{Boolsche Ausdrücke}}

\subsection{Verknüpfungen}
\begin{aufgabe} \noindent
Implementiere eine Funktion \mintinline{python}{set_alarm}, die zurückgibt, ob ein Wecker gestellt werden muss. Die Funktion hat zwei Parameter vom Typ \emph{Bool} (die Typannotation ist \mintinline{python}{bool}). Der erste Parameter gibt an ob eine Person angestellt ist. Der zweite Parameter gibt an, ob die Person gerade Urlaub hat. Die Funktion soll genau dann \mintinline{python}{True} zurückgeben, wenn die Person angestellt ist und gerade keinen Urlaub hat. Ansonsten muss Sie sich keinen Wecker stellen. 


\noindent\url{https://www.codewars.com/kata/568dcc3c7f12767a62000038/train/python}\\
\end{aufgabe}

\begin{pyconsole}
set_alarm(True, False)
set_alarm(True, True)
\end{pyconsole}

\begin{aufgabe} \noindent
Implementiere eine Funktion \mintinline{python}{is_triangle} die  drei Parameter vom Typ \emph{Integer} hat und zurückgibt, ob es ein Dreieck mit diesen Seitenlängen gibt.\\ \hinweis{Jede Seite ist eines Dreicks ist kürzer als die beiden anderen Seiten zusammen}
	
\begin{pyconsole}
is_triangle(1, 1, 30)
is_triangle(1, 2, 3)
is_triangle(5, 5 ,7)
\end{pyconsole}
	
\noindent\url{https://www.codewars.com/kata/56606694ec01347ce800001b/train/python}
\end{aufgabe}



\begin{aufgabe} \noindent
Implementiere eine Funktion \mintinline{python}{xor} die  zwei Parameter vom Typ \emph{Bool} hat und zurückgibt, ob genau ein Argument \mintinline{python}{True} ist.

\begin{pyconsole}
xor(True, True)
xor(True, False)
xor(False, False)
\end{pyconsole}

\noindent\url{https://www.codewars.com/kata/56fa3c5ce4d45d2a52001b3c/train/python}\\
\end{aufgabe}


\hinweis{Ab hier brauchst du die Integer-Division. Lies das entsprechende Kapitel im Skript!}

\begin{aufgabe} \noindent
Implementiere eine Funktion \mintinline{python}{check_for_factor}, die zwei ganze Zahl übergeben bekommt und zurückgibt, ob die zweite Zahl ein Teiler der ersten Zahl ist.
	
\begin{pyconsole}
check_for_factor(12, 3)
check_for_factor(11, 2)
\end{pyconsole}
	
\noindent\url{https://www.codewars.com/kata/55cbc3586671f6aa070000fb/train/python}	
\end{aufgabe}






\end{document}
