\documentclass[class=scrartcl, crop=false]{standalone}
\usepackage[
typ={ohne},
fach=Informatik,
lerngruppe=SG-J1,
%loesungen=seite,
%lizenz={cc-by-nc-sa-4}
module={Aufgaben},
farbig
]{schule}
%\usepackage{polyglossia}
%\setmainlanguage{ngerman}
\usepackage[subpreambles=true]{standalone}
\usepackage{import}
\usepackage{fourier-otf}
\setmonofont{Ubuntu Mono Regular}[Scale=0.9]
\usepackage{shellesc}
\ShellEscape{pythontex \jobname.pytxcode }
\usepackage[
%prettyprinter=pygments, %pygopt={style=emacs}
]{pythontex}


\usepackage{hyperref}


\usepackage{booktabs}

\usepackage{minted}





\newcommand{\expandpyconc}[1]{\expandafter\reallyexpandpyconc\expandafter{#1}}
\newcommand{\reallyexpandpyconc}[1]{\pyconc{exec(compile(open('#1', 'rb').read(), '#1', 'exec'))}}

\newenvironment{pyconcodeblck}[1]
{\newcommand{\snippetfile}{snippet-#1.py}
	\VerbatimEnvironment
	\begin{VerbatimOut}{\snippetfile}}
	{\end{VerbatimOut}
	\expandpyconc{\snippetfile}}

\newcommand{\typesetcode}[1]{\inputpygments{python}{snippet-#1.py}}



\begin{document}

\begin{pyconcodeblck}{for_schleifen}
def print_squares_0_10():
    for i in range(11):
        print(i * i)
        # print(str(i) + " * " + str(i)  + " = " + str(i * i))


def print_squares(n):
    for i in range(n):
        print(i * i)
        # print(str(i) + " * " + str(i)  + " = " + str(i * i))


def print_squares_nicer(n):
    print("The first " + str(n) + " squares are:")
    for i in range(n):
        print(i * i)
        # print(str(i) + " * " + str(i)  + " = " + str(i * i))
    print("Goodbye")


def print_squares_start_end(lower, upper):
    print("The square of the numbers from " + str(lower) + " up to " + str(upper) + " are:")
    for i in range(lower, upper + 1):
        print(i * i)
        # print(str(i) + " * " + str(i)  + " = " + str(i * i))
    print("Goodbye")

def print_squares_start_end_pretty(lower, upper):
    print("The square of the numbers from " + str(lower) + " up to " + str(upper) + " are:")
    for i in range(lower, upper + 1):
        # print(i * i)
        print(str(i) + " * " + str(i)  + " = " + str(i * i))
    print("Goodbye")



def summation(n):
    acc = 0
    for i in range(n + 1):
        acc = acc + i
    return acc

def factorial(n):
    acc = 1 
    for i in range(1, n + 1):
        acc = acc * i
    return acc

def number_to_pwr(number, p): 
    acc = 1
    for i in range(p):
        acc = acc * number
    return acc


def count_sheep(n):
    acc = "" 
    for i in range(1, n + 1):
        acc = acc + str(i) + " sheep..."
    return acc

def sum_cubes(n):
    acc = 0
    for i in range(n + 1):
        acc = acc + i * i * i
    return acc

def get_sum(a,b):
    acc = 0
    if a > b:
        left = b
        right = a
    else:
        left = a
        right = b
    for i in range(left, right + 1):
        acc = acc + i
    return acc

def choose(n,k):
    if k > n:
        return 0
    return factorial(n)//(factorial (n-k) * factorial (k))



def solution(number):
    acc = 0
    for i in range(number):
        if i % 3 == 0 or i % 5 == 0:
            acc  += i
    return acc


def divisors(n):
    acc = 0
    for i in range(1, n + 1):
        if n % i == 0:
            acc = acc + 1
    return acc

def is_prime(n):
    return divisors(n) == 2


def sum_divisors(n):
    acc = 0
    for i in range(1,n):
        if n % i == 0:
            acc = acc + i
    return acc

def is_perfect(n):
    return sum_divisors(n) == n

def amicable_numbers(n1,n2):
    return n1 != n2 and sum_divisors(n1) == n2 and  sum_divisors(n2) == n1


def abundant_number(n):
    return sum_divisors(n) > n

def nth_fib(n):
    a, b = 0, 1
    for i in range(1, n):
        a, b = b , a + b 
    return a
\end{pyconcodeblck}

\section{For-Schleifen}

\subsection{Wiederholte Ausgabe}

\begin{aufgabe} \noindent
Implementiere eine Funktion \mintinline{python}{print_squares_0_10}. Sie gibt die Quadrate der Zahlen von $0$ bis $10$ an der Konsole aus. 

\begin{pyconsole}
print_squares_0_10()
\end{pyconsole}

\end{aufgabe}



\begin{aufgabe} \noindent
Implementiere eine Funktion \mintinline{python}{print_squares}. Dieser wird eine positive Zahl $n$ übergeben. Sie gibt die ersten $n$ Quadratzahlen an der Konsole aus. 

\begin{pyconsole}
print_squares(5)
\end{pyconsole}

\end{aufgabe}




\begin{aufgabe} \noindent
Implementiere eine Funktion \mintinline{python}{print_squares_nicer}. Dieser wird eine positive Zahl $n$ übergeben. Sie kündigt zuerst an, wie viele Quadratzahlen sie  ausgeben wird. Nach der Ausgabe dieser Quadratzahlen verabschiedet sie sich von dem Benutzer.
\begin{pyconsole}
print_squares_nicer(5)
\end{pyconsole}

\end{aufgabe}


\begin{aufgabe} \noindent
Implementiere eine Funktion \mintinline{python}{print_squares_start_end}. Dieser werden zwei  positive Zahlen $m$ und $n$ übergeben. Sie gibt die Quadrate der Zahlen von $m$ bis $n$ aus. Die Funktion gibt zunächst an welche Quadratzahlen sie  ausgibt. Nach der Ausgabe dieser Quadratzahlen verabschiedet sie sich von dem Benutzer.

\begin{pyconsole}
print_squares_start_end(5, 7)
\end{pyconsole}

\end{aufgabe}

\begin{aufgabe}
Erweitere die Funktion aus der letzten Aufgabe zu einen Funktion \mintinline{python}{print_squares_start_end_pretty} so, dass in jedem Schritt auch die Rechnung angezeigt wird.
\end{aufgabe}

\begin{pyconsole}
print_squares_start_end_pretty(5, 7)
\end{pyconsole}

\subsection{Akkumulator-Pattern}

\begin{aufgabe} \noindent
Implementiere eine Funktion \mintinline{python}{summation}. Dieser wird eine positive Zahl $n$ übergeben. Sie berechnet die Summe der Zahlen von $1$ bis $n$. Es gilt z.B. $ \mathtt{summation} (\mathtt{3}) = \mathtt{1} + \mathtt{2} + \mathtt{3}$ und allgemein: $ \mathtt{summation} (\mathtt{n}) =  \mathtt{1} + \dots \mathtt{n}$.

\begin{pyconsole}
summation(1)
summation(3)
summation(5)
\end{pyconsole}

\noindent\url{https://www.codewars.com/kata/55d24f55d7dd296eb9000030/train/python}

\end{aufgabe}


\begin{aufgabe} \noindent
Implementiere eine Funktion \mintinline{python}{number_to_pwr}, mit der Potenzen berechnet werden können. Der Funktion wird die Basis und die Hochzahl übergeben. Sie gibt die berechnete Potenz zurück. Beispiel:
$$ \mathtt{number\_to\_pwr} (\mathtt{2, 3}) =  \mathtt{2 \cdot 2  \cdot 2}$$

\begin{pyconsole}
number_to_pwr(3, 1)
number_to_pwr(2, 3)
number_to_pwr(3, 3)
\end{pyconsole}

\noindent\url{https://www.codewars.com/kata/562926c855ca9fdc4800005b/train/python}

\end{aufgabe}



\begin{aufgabe} \noindent
Implementiere eine Funktion \mintinline{python}{factorial}. Dieser wird eine positive Zahl $n$ übergeben. Sie berechnet das Produkt der Zahlen von $1$ bis $n$. Es gilt z.B. $ \mathtt{factorial} (\mathtt{3}) = \mathtt{1} * \mathtt{2} *  \mathtt{3}$ und allgemein: $ \mathtt{factorial} (\mathtt{n}) =  \mathtt{1} * \dots \mathtt{n}$.
Das leere Produkt (mit null Faktoren) ergibt Eins.

\begin{pyconsole}
factorial(0)
factorial(1)
factorial(3)
\end{pyconsole}

\noindent\url{https://www.codewars.com/kata/57a049e253ba33ac5e000212/train/python}

\end{aufgabe}

\begin{aufgabe} \noindent
Implementiere eine Funktion \mintinline{python}{choose}. Dieser werden zwei ganze Zahlen $N$ und $k$ übergeben. 
Sie gibt zurück, wie viele Möglichkeiten es gibt $k$ Elemente  aus $N$ Elementen auszuwählen. 
Solange $k$ nicht größer ist als $N$ kann man für die Berechnung Binomialkoeffizienten nutzen.

$$\binom{N}{k} = \frac{n!}{k! \cdot (n - k)!}$$

\begin{pyconsole}
choose(1, 1)
choose(3, 2)
choose(5, 10)
\end{pyconsole}

\noindent\url{https://www.codewars.com/kata/55b22ef242ad87345c0000b2/train/python}

\end{aufgabe}


\begin{aufgabe} \noindent
Implementiere eine Funktion \mintinline{python}{count_sheep}, die beim Einschlafen hilft. Dieser wird eine natürliche Zahl $n$ übergeben. Sie gibt einen String zurück in dem entsprechend viele Schafe gezählt werden.

\begin{pyconsole}
count_sheep(0)
count_sheep(1)
count_sheep(3)
\end{pyconsole}

\noindent\url{https://www.codewars.com/kata/5b077ebdaf15be5c7f000077/train/python}

\end{aufgabe}



\begin{aufgabe} \noindent
Implementiere eine Funktion \mintinline{python}{sum_cubes}. Dieser wird eine positive Zahl $n$ übergeben. Sie berechnet Summe der ersten $n$ Kubikzahlen. Es gilt z.B. $ \mathtt{sum\_cubes} (\mathtt{3}) = \mathtt{1^{3} + 2 ^{3} + 3^{3}}$ und allgemein: $ \mathtt{sum\_cubes} (\mathtt{n}) =  \mathtt{1^{3}} + \dots \mathtt{n^{3}}$.


\begin{pyconsole}
sum_cubes(1)
sum_cubes(2)
sum_cubes(3)
\end{pyconsole}

\noindent\url{https://www.codewars.com/kata/59a8570b570190d313000037/train/python}

\end{aufgabe}


\begin{aufgabe} \noindent
Implementiere eine Funktion \mintinline{python}{get_sum}. Dieser werden zwei ganze Zahlen übergeben. Sie gibt die Summe aller Zahlen dazwischen einschließlich der beiden Zahlen zurück. Es gilt z.B. $ \mathtt{get\_sum} (\mathtt{3, 1}) = \mathtt{1 + 2 + 3}$


\begin{pyconsole}
get_sum(1, 2)
get_sum(3, 1)
\end{pyconsole}

\noindent\url{https://www.codewars.com/kata/55f2b110f61eb01779000053/train/python}

\end{aufgabe}

\begin{aufgabe} \noindent
Nutze  \mintinline{python}{for}-Schleifen um eine Funktion \mintinline{python}{nth_fib} zu schreiben, mit der die $n$-te Fibonacci-Zahl berechnet werden kann. Die ersten beiden Fibonacci-Zahlen sind $0$ und $1$. Jede weitere Fibonacci-Zahl ist die Summe ihrer beiden Vorgänger.

\begin{pyconsole}
nth_fib(1)
nth_fib(2)
nth_fib(3)
nth_fib(4)
\end{pyconsole}

\noindent\url{https://www.codewars.com/kata/522551eee9abb932420004a0/train/python}

\end{aufgabe}


\Huge{\hinweis{Für die folgenden Aufgaben musst du dich mit Integer-Division auskennen!}}

\normalsize
\begin{aufgabe} \noindent
Implementiere eine Funktion \mintinline{python}{solution}. Dieser wird eine ganze Zahl $n$ übergeben. 
Sie gibt die Summe der Vielfache von $3$ oder $5$, die kleiner als diese Zahl sind, zurück.

\begin{pyconsole}
solution(6)
solution(3)
\end{pyconsole}

\noindent\url{https://www.codewars.com/kata/514b92a657cdc65150000006/train/python}

\end{aufgabe}


\begin{aufgabe} \noindent
Implementiere eine Funktion \mintinline{python}{divisors}. Dieser wird eine ganze Zahl $n$ übergeben. 
Sie gibt die Anzahl der Teiler dieser Zahl zurück. Die Zahl $4$ hat die Teiler $1, 2$ und $4$. Also gilt $\mathtt{divisors(4) = 3}$.

\begin{pyconsole}
divisors(4)
divisors(5)
\end{pyconsole}

\noindent\url{https://www.codewars.com/kata/542c0f198e077084c0000c2e/train/python}

\end{aufgabe}

\begin{aufgabe} \noindent
Eine Primzahl ist eine natürliche Zahl größer Eins, die nur durch eins und sich selbst teilbar ist. 
Implementiere eine Funktion \mintinline{python}{is_prime}. Dieser wird eine ganze Zahl $n$ übergeben. 
Sie gibt die zurück, ob die Zahl eine Primzahl ist.

\begin{pyconsole}
is_prime(6)
is_prime(7)
\end{pyconsole}

\noindent\url{https://www.codewars.com/kata/53daa9e5af55c184db00025f/train/python}

\end{aufgabe}





\begin{aufgabe} \noindent
Eine natürliche Zahl ist perfekt, wenn die Summe ihrer echten Teiler (bis auf sie selbst) genau die Zahl ergibt.
Z.B. gilt $1 + 2 + 3 = 6$. Also ist $6$ eine perfekte Zahl.
Implementiere eine Funktion \mintinline{python}{is_perfect}. Dieser wird eine ganze Zahl $n$ übergeben. 
Sie gibt die zurück, ob die Zahl perfekt ist.

\begin{pyconsole}
is_perfect(6)
is_perfect(7)
\end{pyconsole}

\noindent\url{https://www.codewars.com/kata/56a28c30d7eb6acef700004d/train/python}

\end{aufgabe}


\begin{aufgabe} \noindent
Eine natürliche Zahl ist abundant(überladen), wenn die Summe ihrer echten Teiler (bis auf sie selbst) größer als die Zahl selbst ist. Z.B. ist $12$ eine abundante Zahl, da die echten Teiler von $12$ die Zahlen  $ 1, 2, 3,  4$ und $6$ sind. Deren Summe  $16$ ist größer als $12$. 

Implementiere eine Funktion \mintinline{python}{abundant_number}, die prüft, ob eine natürliche Zahl abundant ist. 
\begin{pyconsole}
abundant_number(12)
abundant_number(7)
\end{pyconsole}

\noindent\url{https://www.codewars.com/kata/56a75b91688b49ad94000015/train/python}

\end{aufgabe}




\begin{aufgabe} \noindent
Zwei natürliche Zahl sind befreundet, wenn sie verschieden sind und die Summe ihrer Teiler jeweils der anderen Zahl entspricht.
Z.B. sind $220$ und $284$ befreundet.  Die Teiler von $220$ sind $ 1, 2, 4, 5, 10, 11, 20, 22, 44, 55$ und $110$. Die Summe dieser Zahlen ist  $284$. 

Die Teiler von $284$ sind $ 1, 2, 4, 71  $ und $142$. Deren Summe ist $220$.
Implementiere eine Funktion \mintinline{python}{amicable_numbers}, die prüft ob zwei Zahlen befreundet sind.

\begin{pyconsole}
amicable_numbers(220, 284)
amicable_numbers(10, 11)
\end{pyconsole}

\noindent\url{https://www.codewars.com/kata/56b5ebaa26fd54188b000018/train/python}

\end{aufgabe}




\begin{aufgabe} \noindent
Implementiere eine schnellere Version von \mintinline{python}{is_prime}. Überlege dir dafür wie groß die echten Teiler einer Zahl maximal sein können.

\noindent\url{https://www.codewars.com/kata/5262119038c0985a5b00029f/train/python}
\end{aufgabe}




\end{document}
