\documentclass[class=scrartcl, crop=false]{standalone}
\usepackage[
typ={ohne},
fach=Informatik,
lerngruppe=SG-J1,
%loesungen=seite,
%lizenz={cc-by-nc-sa-4}
module={Aufgaben},
farbig
]{schule}
%\usepackage{polyglossia}
%\setmainlanguage{ngerman}
\usepackage[subpreambles=true]{standalone}
\usepackage{import}
\usepackage{fourier-otf}
\setmonofont{Ubuntu Mono Regular}[Scale=0.9]
\usepackage{shellesc}
\ShellEscape{pythontex \jobname.pytxcode }
\usepackage[
%prettyprinter=pygments, %pygopt={style=emacs}
]{pythontex}


\usepackage{hyperref}


\usepackage{booktabs}

\usepackage{minted}





\newcommand{\expandpyconc}[1]{\expandafter\reallyexpandpyconc\expandafter{#1}}
\newcommand{\reallyexpandpyconc}[1]{\pyconc{exec(compile(open('#1', 'rb').read(), '#1', 'exec'))}}

\newenvironment{pyconcodeblck}[1]
{\newcommand{\snippetfile}{snippet-#1.py}
	\VerbatimEnvironment
	\begin{VerbatimOut}{\snippetfile}}
	{\end{VerbatimOut}
	\expandpyconc{\snippetfile}}

\newcommand{\typesetcode}[1]{\inputpygments{python}{snippet-#1.py}}


\begin{document}
\begin{pyconcodeblck}{strings_als_container}
def no_space(s):
    acc = ""
    for char in s:
        if char != " ":
            acc = acc + char
    return acc

def contamination(text, new_char):
    acc = ""
    for char in text:
        acc = acc + new_char
    return acc

def double_char(s):
    acc = ""
    for char in s:
        acc = acc + 2 * char
    return acc

def is_vowel(char):
    return char == "a" or char == "e" or char == "i" or char == "o" or char == "u" or char == "A" or char == "E" or char == "I" or char == "O" or char == "U"


def disemvowel(s):
    acc = ""
    for char in s:
        if not is_vowel(char):
            acc = acc + char
    return acc

def move_vowels(input): 
    not_vowels = ""
    vowels =""
    for char in input:
        if char == "a" or char == "e" or char == "i" or char == "o" or char == "u":
            vowels = vowels + char
        else:
            not_vowels = not_vowels + char
    return not_vowels + vowels
            
def sum_digits(number):
    acc = 0
    for digit in str(number):
        if digit != "-":
            acc = acc + int(digit)
    return acc
                    
\end{pyconcodeblck}
\section{Strings als Container}
\begin{aufgabe}
Schreibe eine Funktion \mintinline{python}|contamination|. Dieser werden ein String und ein einzelner Buchstabe als String übergeben. Sie gibt einen String zurück, in dem alle Buchstaben im ursprünglichen String durch den übergebenen Buchstaben ersetzt wurden.

\begin{pyconsole}
contamination("fadsf  adfasdf", "-")
contamination("  fa adsf  adfasdf", "a")
\end{pyconsole}


\url{https://www.codewars.com/kata/596fba44963025c878000039/train/python}
\end{aufgabe}



\begin{aufgabe}
Schreibe eine Funktion \mintinline{python}|double_char|. Dieser wird ein String übergeben. Sie gibt einen String zurück, in dem alle Buchstaben im ursprünglichen String verdoppelt wurden.

\begin{pyconsole}
double_char("Hello")
double_char("Bye")
\end{pyconsole}


\url{https://www.codewars.com/kata/56b1f01c247c01db92000076/train/python}
\end{aufgabe}




\begin{aufgabe}
Schreibe eine Funktion \mintinline{python}|nospace|. Dieser wird ein String übergeben. Sie gibt den String ohne Leerzeichen zurück.

\begin{pyconsole}
no_space("fadsf  adfasdf")
no_space("  fa adsf  adfasdf")
\end{pyconsole}


\url{https://www.codewars.com/kata/57eae20f5500ad98e50002c5/train/python}
\end{aufgabe}


\begin{aufgabe}
Schreibe eine Funktion \mintinline{python}|disemvowel|. Dieser wird ein String übergeben. Sie gibt den String ohne Vokale zurück.

\begin{pyconsole}
disemvowel("Day")
disemvowel("Apple")
disemvowel("Hello World")
\end{pyconsole}


\url{https://www.codewars.com/kata/56bf3287b5106eb10f000899/train/python}
\end{aufgabe}


\begin{aufgabe}
Schreibe eine Funktion \mintinline{python}|move_vowels|. Dieser wird ein String übergeben, der nur Kleinbuchstaben enthält. Sie gibt den String zurück, indem alle Vokale des Arguments nach hinten verschoben wurden.

\begin{pyconsole}
move_vowels("day")
move_vowels("apple")
move_vowels("peace")
\end{pyconsole}


\url{https://www.codewars.com/kata/56bf3287b5106eb10f000899/train/python}
\end{aufgabe}



\begin{aufgabe}
Schreibe eine Funktion \mintinline{python}|sum_digits|. Dieser wird eine Zahl als Integer übergeben. Sie gibt die Summe der Ziffern in der Zahl zurück.
\begin{pyconsole}
sum_digits(123)
sum_digits(-51)
\end{pyconsole}


\url{https://www.codewars.com/kata/52f3149496de55aded000410/train/python}
\end{aufgabe}
\end{document}
